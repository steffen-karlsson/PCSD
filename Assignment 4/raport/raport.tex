\documentclass[10pt,a4paper]{article}
\usepackage[T1]{fontenc}
\usepackage[utf8]{inputenc}
\usepackage{amsmath}
\usepackage{amsfonts}
\usepackage{amssymb}
\usepackage{listings}
\usepackage{graphicx}
\newcommand{\folge}[1]{\left \lbrace #1 \right \rbrace }
\lstset{language=Java, numbers=left, numberstyle=\footnotesize}
\author{Thorbjørn Christensen \\
Steffen Karlsson \\
Kai Ejler Rasmussen}
\title{Principles of Computer System Design - Assignment 4}
\begin{document}
\maketitle

\section*{Exercises}
\subsection*{Question 1: Reliability}
\begin{enumerate}
	\item In the daisy chain network both network links must work for the network to be fully connected, and the probability that one of them fails is $2p$ (we assume that the event of failing links are independent), so we get the probability that all buildings are connected to be: \\
  $P(connected) = 1 - 2p$
 	\item In the fully connected network the network we still connect all the buildings if we loose a single link, but the connnection will be lost if we loose two or more links. We get the probability (again assuming that all failures are independent) that all buildings are connected to be: \\
  $P(connect) = 1 - (3 \cdot p^2 \cdot (1-p) + p^3)$. \\
  $3 \cdot p^2 \cdot (1-p)$ is the prohability that exactly two links fails (using the bionomial probability density function), and $p^3$ is the prohability that all three links fails. 
  \item The prohability that the daisy chain network is connecting all buldings: \\
  $C_1 = 1 - 2 \cdot 0.000001 = 0.999998$ \\
  The prohability that the fully connected network is connecting all buildings: \\
  $C_2 = 1 - (3 \cdot (0.0001)^2 \cdot (1-0.0001) + 0.0001^3) = 0.99999996$ \\
  So assuming that all network failures are independent and that the town council wants the most reliable solution he should go with the fully connected network.
\end{enumerate}


\subsection*{Question 2: Distributed Coordination}
When a coordinator receives a transaction request, using the three-phase commit protocol, the coordinator sends a canCommit message to the participant and moves along to a waiting/idle state, unless a failure happens - if such scenario, the coordinator aborts. 
\newline

When the participant receives the canCommit message, it sends yes or no back to the coordinator depending on if it agrees or not - or no if a failure happens followed by moving to the aborted state. 
\newline

During the waiting/idle state on the coordinator, if anything unexpected, such as a failure or timeout happens the coordinator sends an abort message to all participants. A timeout could happens, because the coordinator usually implements some kind of timing mechanism, such that if no answer is received from a participant within a certain amount of time, the coordinate assumes its a no, which is how the coordinator deals with delays during the 'uncertain' periods.
\newline

When or if the coordinator receives an yes message from all participants in the waiting state, it responds with a preCommit message and moves to the next state. In this state all the same criteria are applicable as for the previous state, regarding timeout.
\newline

When the participant receives the preCommit message, it sends an acknowledgement back to the coordinator, otherwise it aborts the transaction. If the coordinator receives an acknowledgement from all participants within their timeout, it sends a doCommit message to the participant, which is answered with a haveCommited message if all is completed, either it aborts.



\end{document}